\documentclass[10pt]{exam}
\usepackage[T1]{fontenc}
\usepackage[paper=a4paper,margin=2cm]{geometry}
\usepackage[sfdefault,light]{roboto}

\usepackage[usenames,dvipsnames]{xcolor}
\usepackage{amsmath,amssymb,array,graphicx,enumitem,listings,lstautogobble,multicol,textcomp,titlesec}
\usepackage{mathtools}

\setlength\parindent{0cm}

%Title & Section Formatting
\titleformat{\section}{\normalfont\Large\bfseries}{}{0em}{}[{\titlerule[0.5pt]}]
\titleformat{\subsection}{\normalfont\large\bfseries}{}{0em}{}

%Mathematical Shortcuts
\newcommand{\floor}[1]{\left\lfloor #1 \right\rfloor}
\newcommand{\ceil}[1]{\left\lceil #1 \right\rceil}

%Listings Shortcuts & Settings
\newcommand{\code}[1]{\lstinline{#1}}
\lstset{language=Java,
        autogobble=true,
        basicstyle=\ttfamily,
        commentstyle=\color{black!45},
        keywordstyle=\bfseries,
        showstringspaces=false,
        upquote=true}

%General Shortcuts
\newcommand{\blankpage}{\null\thispagestyle{empty}\addtocounter{page}{-1}\newpage}


%Color-Coded questions
\newcommand{\ColourQuestion}[3]{\renewcommand{\questionlabel}{\colorbox{#1}{\bfseries\color{white}\thequestion}\hfill}\question[#3] #2}
\newcommand{\BlueQuestion}[1]{\ColourQuestion{RoyalBlue}{#1}{3}}
\newcommand{\GreenQuestion}[1]{\ColourQuestion{ForestGreen}{#1}{5}}
\newcommand{\YellowQuestion}[1]{\ColourQuestion{Goldenrod}{#1}{10}}
\newcommand{\RedQuestion}[1]{\ColourQuestion{BrickRed}{#1}{20}}
\newcommand{\PurpleQuestion}[1]{\ColourQuestion{RoyalPurple}{#1}{30}}
\pointsinrightmargin

\header{\footnotesize\scshape Assignment \#\AssignmentNumber: \AssignmentTitle}{}{}
\cfoot{\footnotesize\scshape \AssignmentCourse\\Woodstock School---Mussoorie, Uttarakhand---India}


\def\AssignmentCourse{AP Computer Science A}
\def\AssignmentNumber{07}
\def\AssignmentTitle{Classes \& Objects}

\begin{document}
  \begin{questions}
    \BlueQuestion{Explain why the following implemention is incorrect for a constructor of the \code{Car} class.}
    \begin{lstlisting}
      public Car {
        private int nDoors;   // How many doors does the car have?
        private Color color;  // What color is the car?

        public void Car(int doors, Color color) {
          nDoors = doors;
          color = color;
        }

        /* Additional attributes and methods not shown. */
      }
    \end{lstlisting}

    %Object Creation/Data Access
    \BlueQuestion{What happens when the following method is run using the \code{Car} class from Question \#1. (You may assume all problems with the class constructor have been resolved.)}
    \begin{lstlisting}
      public static void main(String[] args) {
        // Create a new red, four-door car.
        Car car = new Car(4, Color.RED);

        // Paint the car blue!
        car.color = Color.BLUE;
      }
    \end{lstlisting}

    \GreenQuestion{Java requires that all programming code be part of a class. Some other object-oriented programming languages, such as C++, allow for ``global'' variables and methods; that is, C++ allows for programming code to exist outside of container classes. In your own words, what might be some benefits/drawbacks to Java's approach?}

    %Explanation of static vs. non-static methods (5 questions: Select `static` or `non-static`)
    \GreenQuestion{For each of the following examples, indicate whether the described method or attribute should be \code{static} or non-\code{static}.}
    \begin{enumerate}[label=\texttt{\alph*.}]
      \item A method of the \code{Employee} class that increases an employee's salary by $10\%$.
      \item An attribute of the \code{Cat} class that contains a cat's genus.
      \item A method of the \code{Television} class that changes a TVs channel.
      \item A method of the \code{Algebra} class that solves a given (i.e., as a parameter) polynomial equation.
      \item An attribute of the \code{Person} class that contains a person's height.
    \end{enumerate}

    %Basic Application of Classes & Objects: Implementation from UML Diagram
    \YellowQuestion{Implement the following class in Java. Ensure that the visibility of all attributes are \code{private} and all appropriate accessor methods are implemented.}
    \begin{center}
      \includegraphics[scale=0.75]{graphics/assignments/apcsa_07_AnimalUML}
    \end{center}

    \pagebreak
    %Application of Classes & Objects: `Point`, `Line`, `Geometry`
    \RedQuestion{Create the \code{Point}, \code{Line}, and \code{Geometry} classes with the following specifications.}

    \code{Point}:
    \begin{itemize}
      \item Contains two attributes: \code{x} and \code{y} to represent the rectangular coordinates of the point.
      \item Contains the appropriate constructor for a given set of coordinates.
      \item Contains the method: \code{distanceTo(Point p)} that will calculate the distance to the passed point, \code{p}.
    \end{itemize}

    \code{Line}:
    \begin{itemize}
      \item Contains two attributes: \code{m} and \code{b} to represent the slope and y-intercept of the line.\\
      {\small\textbf{Note:} \code{b} should be a \code{Point} object.}
      \item Contains an overloaded constructor for each of the following possible set of given values:
      \begin{itemize}
        \item Slope, y-Intercept
        \item Slope, Any point
        \item Any two points
      \end{itemize}
      \item Contains the method: \code{contains(Point p)} that will return \code{true} if the line contains the point, \code{p}, and \code{false} otherwise.
    \end{itemize}

    \code{Geometry}:
    \begin{itemize}
      \item Contains the method: \code{distance(Point p1, Point p2)} that will return the distance between the two given points.
      \item Contains the method: \code{midpoint(Point p1, Point p2)} that will return a \code{Point} that is the midpoint between the two given points.
      \item Contains the method: \code{perpendicularLine(Line l, Point p)} that will return a \code{Line} that is perpendicular to \code{l} that passes through \code{p}.
    \end{itemize}
  \end{questions}
\end{document}
