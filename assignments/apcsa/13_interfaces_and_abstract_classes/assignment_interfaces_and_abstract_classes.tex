\documentclass[10pt]{exam}
\usepackage[T1]{fontenc}
\usepackage[paper=a4paper,margin=2cm]{geometry}
\usepackage[sfdefault,light]{roboto}

\usepackage[usenames,dvipsnames]{xcolor}
\usepackage{amsmath,amssymb,array,graphicx,enumitem,listings,lstautogobble,multicol,textcomp,titlesec}
\usepackage{mathtools}

\setlength\parindent{0cm}

%Title & Section Formatting
\titleformat{\section}{\normalfont\Large\bfseries}{}{0em}{}[{\titlerule[0.5pt]}]
\titleformat{\subsection}{\normalfont\large\bfseries}{}{0em}{}

%Mathematical Shortcuts
\newcommand{\floor}[1]{\left\lfloor #1 \right\rfloor}
\newcommand{\ceil}[1]{\left\lceil #1 \right\rceil}

%Listings Shortcuts & Settings
\newcommand{\code}[1]{\lstinline{#1}}
\lstset{language=Java,
        autogobble=true,
        basicstyle=\ttfamily,
        commentstyle=\color{black!45},
        keywordstyle=\bfseries,
        showstringspaces=false,
        upquote=true}

%General Shortcuts
\newcommand{\blankpage}{\null\thispagestyle{empty}\addtocounter{page}{-1}\newpage}


%Color-Coded questions
\newcommand{\ColourQuestion}[3]{\renewcommand{\questionlabel}{\colorbox{#1}{\bfseries\color{white}\thequestion}\hfill}\question[#3] #2}
\newcommand{\BlueQuestion}[1]{\ColourQuestion{RoyalBlue}{#1}{3}}
\newcommand{\GreenQuestion}[1]{\ColourQuestion{ForestGreen}{#1}{5}}
\newcommand{\YellowQuestion}[1]{\ColourQuestion{Goldenrod}{#1}{10}}
\newcommand{\RedQuestion}[1]{\ColourQuestion{BrickRed}{#1}{20}}
\newcommand{\PurpleQuestion}[1]{\ColourQuestion{RoyalPurple}{#1}{30}}
\pointsinrightmargin

\header{\footnotesize\scshape Assignment \#\AssignmentNumber: \AssignmentTitle}{}{}
\cfoot{\footnotesize\scshape \AssignmentCourse\\Woodstock School---Mussoorie, Uttarakhand---India}


\def\AssignmentCourse{AP Computer Science A}
\def\AssignmentNumber{13}
\def\AssignmentTitle{Interfaces \& Abstract Classes}

\begin{document}
  \begin{questions}
    \BlueQuestion{Suppose we have the following classes in our program.}

      \begin{lstlisting}
        public abstract class Toy { ... }
        public class Doll extends Toy { ... }
        public class Barbie extends Doll { ... }
      \end{lstlisting}

      Assume that \code{Toy}, \code{Doll}, and \code{Barbie} each have constructors that accept no parameters.
      
      If an object, \code{child}, has a method \code{play(Doll doll)}, which of the following statements will compile with no errors?
      
      \begin{choices}
        \choice \code{child.play(new Object());}
        \choice \code{child.play(new Toy());}
        \choice \code{child.play(new Doll());}
        \choice \code{child.play(new Barbie());}
      \end{choices}

    \BlueQuestion{Indicate whether each of the following statements is \code{True} or \code{False.}
      \begin{choices}
        \choice You can create interfaces that are ``subinterfaces'' of other interfaces.
        \choice A class can implement only one interface at a time.
        \choice An interface can be used as the data type for parameters in methods and while declaring variables.
      \end{choices}
    }

    \GreenQuestion{Indicate whether each of the following statements is \code{True} or \code{False}.
      \begin{choices}
        \choice You can't create objects of an abstract class.
        \choice You can't create non-abstract subclasses of an abstract class.
        \choice You can't create abstract subclasses of an abstract class.
        \choice \code{Object} is an abstract class.
        \choice An abstract class can be used as the data type for parameters in methods and while declaring variables.
      \end{choices}
    }

    \GreenQuestion{Complete the \code{compareTo()} method below which will determine whether or not an \code{Atom} object appears before or after another \code{Atom} in the periodic table.}
    
      \begin{lstlisting}
        class Atom implements Comparable<Atom> {
          private int atomicNumber;

          public int compareTo(Atom a) {
            //  To be implemented.
          }

          // There may be instance variables, constructors, and other methods not shown.
        }
      \end{lstlisting}

    \YellowQuestion{Answer each of the following questions.
      \begin{parts}
        \part{Describe a \code{BankTransaction} class which would handle the various different types of transactions between accounts in a bank. In particular, it should be able to describe withdrawals, deposits, and tranfers of funds between accounts. How would you design such a class? In particular, describe the instance variables and methods which might exist for this class.}
        \part{Would it be appropriate to represent \code{BankTransaction} as a \code{class}, \code{abstract class}, or an \code{interface}? Explain your choice.}
      \end{parts}
    }

    \RedQuestion{Complete each of the following exercises and questions.}
      \begin{parts}
        \part{Implement a \code{BankAccount} abstract class with:
          \begin{itemize}
            \item an appropriate instance variable to hold the current balance in the account,
            \item abstract method \code{processTransaction()} which will take a \code{BankTransaction} parameter,
            \item helper methods \code{depositAmount()} and \code{withdrawAmount()} which should update the balance in the account appropriately, and
            \item an accessor method to the balance in the account.
          \end{itemize}
        }
        \pagebreak
        \part{Complete the \code{makePayment()} method below.}\\
        {\small\textbf{Note:} This method would likely be part of a larger system and should \emph{not} be included in the \code{BankAccount} class.}

        \begin{lstlisting}[basicstyle=\ttfamily\small,tabsize=2]
          /**
           * Generates appropriate BankTransaction objects to withdraw money from one 
           * account and deposit it into another.
           * 
           * @param from  The account to transfer an amount from.
           * @param to    The account to transfer an amount to.
           * @param amt   The amount to transfer.
           * @return  Returns true if the transaction was successful and 
           *          false otherwise.
           */
          public boolean makePayment(BankAccount from, BankAccount to, double amt) {
            //  To be implemented
          }
        \end{lstlisting}
        \part{Suppose \code{CheckingAccount}, \code{SavingsAccount} and \code{LoanAccount} were each subclasses of \code{BankAccount}. Briefly describe how their \code{processTransaction()} implementations might differ.}
        \part{Why is it appropriate to implement \code{BankAccount} as an abstract class rather than an interface?}
      \end{parts}
  \end{questions}
\end{document}
