\documentclass[10pt]{exam}
\usepackage[T1]{fontenc}
\usepackage[paper=a4paper,margin=2cm]{geometry}
\usepackage[sfdefault,light]{roboto}

\usepackage[usenames,dvipsnames]{xcolor}
\usepackage{amsmath,amssymb,array,graphicx,enumitem,listings,lstautogobble,multicol,textcomp,titlesec}
\usepackage{mathtools}

\setlength\parindent{0cm}

%Title & Section Formatting
\titleformat{\section}{\normalfont\Large\bfseries}{}{0em}{}[{\titlerule[0.5pt]}]
\titleformat{\subsection}{\normalfont\large\bfseries}{}{0em}{}

%Mathematical Shortcuts
\newcommand{\floor}[1]{\left\lfloor #1 \right\rfloor}
\newcommand{\ceil}[1]{\left\lceil #1 \right\rceil}

%Listings Shortcuts & Settings
\newcommand{\code}[1]{\lstinline{#1}}
\lstset{language=Java,
        autogobble=true,
        basicstyle=\ttfamily,
        commentstyle=\color{black!45},
        keywordstyle=\bfseries,
        showstringspaces=false,
        upquote=true}

%General Shortcuts
\newcommand{\blankpage}{\null\thispagestyle{empty}\addtocounter{page}{-1}\newpage}


%Color-Coded questions
\newcommand{\ColourQuestion}[3]{\renewcommand{\questionlabel}{\colorbox{#1}{\bfseries\color{white}\thequestion}\hfill}\question[#3] #2}
\newcommand{\BlueQuestion}[1]{\ColourQuestion{RoyalBlue}{#1}{3}}
\newcommand{\GreenQuestion}[1]{\ColourQuestion{ForestGreen}{#1}{5}}
\newcommand{\YellowQuestion}[1]{\ColourQuestion{Goldenrod}{#1}{10}}
\newcommand{\RedQuestion}[1]{\ColourQuestion{BrickRed}{#1}{20}}
\newcommand{\PurpleQuestion}[1]{\ColourQuestion{RoyalPurple}{#1}{30}}
\pointsinrightmargin

\header{\footnotesize\scshape Assignment \#\AssignmentNumber: \AssignmentTitle}{}{}
\cfoot{\footnotesize\scshape \AssignmentCourse\\Woodstock School---Mussoorie, Uttarakhand---India}


\def\AssignmentCourse{AP Computer Science A}
\def\AssignmentNumber{15}
\def\AssignmentTitle{Recursive Functions}

\begin{document}
  \begin{questions}
    \BlueQuestion{Explain what happens when a negative value is passed to the \code{Factorial()} method programmed in class.}

    \BlueQuestion{Modify \code{Factorial()} in order to handle the error encountered in Question \#1 elegantly.}

    \GreenQuestion{Evaluate \code{mystery(2, 25)} and \code{mystery(3, 11)} for the following recursive function.}
      \begin{lstlisting}
        public static int mystery(int a, int b) {
          if (b == 0)
            return 0;
          if (b % 2 == 0)
            return mystery(a + a, b / 2);
          return mystery(a + a, b / 2) + a;
        }
      \end{lstlisting}

    \GreenQuestion{Explain what is wrong with the following recursive function.}
      \begin{lstlisting}
        public static String work(int n) {
          String s = work(n - 3) + n + work(n - 2) + n;
          if (n <= 0)
            return "";
          return s;
        }
      \end{lstlisting}

    \YellowQuestion{\emph{Quicksort.} Implement the \emph{Quicksort Algorithm} as detailed below.}
      \textbf{Input:} An array, \code{A}, of integer values.

      \begin{enumerate}
        \item Pick any element from \code{A} to act as the \code{pivot} value.
        \item Reorder the array so that all elements in \code{A} less than or equal to \code{pivot} are in a lower index than \code{pivot} and all elements in \code{A} greater than or equal to \code{pivot} are in a higher index than \code{pivot}.
        \item Recursively apply steps \#1 \& 2 on the subarray of elements less than \code{pivot} and on the subarray of elements greater than \code{pivot}.
      \end{enumerate}

      \textbf{Output:} An array, \code{A}, of integer values in ascending order.

    \RedQuestion{\emph{Partitions.} A \emph{partition} of a positive integer, $n$, is its representation as a sum of positive integers, $n = p_1 + p_2 + \cdots + p_k$. Write a method that prints out all possible partitions of a given positive integer, $n$. Consider only partitions where $p_1 \leq p_2 \leq \cdots \leq p_k$.}
  \end{questions}
\end{document}
