\documentclass[10pt]{exam}
\usepackage[T1]{fontenc}
\usepackage[paper=a4paper,margin=2cm]{geometry}
\usepackage[sfdefault,light]{roboto}

\usepackage[usenames,dvipsnames]{xcolor}
\usepackage{amsmath,amssymb,array,graphicx,enumitem,listings,lstautogobble,multicol,textcomp,titlesec}
\usepackage{mathtools}

\setlength\parindent{0cm}

%Title & Section Formatting
\titleformat{\section}{\normalfont\Large\bfseries}{}{0em}{}[{\titlerule[0.5pt]}]
\titleformat{\subsection}{\normalfont\large\bfseries}{}{0em}{}

%Mathematical Shortcuts
\newcommand{\floor}[1]{\left\lfloor #1 \right\rfloor}
\newcommand{\ceil}[1]{\left\lceil #1 \right\rceil}

%Listings Shortcuts & Settings
\newcommand{\code}[1]{\lstinline{#1}}
\lstset{language=Java,
        autogobble=true,
        basicstyle=\ttfamily,
        commentstyle=\color{black!45},
        keywordstyle=\bfseries,
        showstringspaces=false,
        upquote=true}

%General Shortcuts
\newcommand{\blankpage}{\null\thispagestyle{empty}\addtocounter{page}{-1}\newpage}

\usepackage{draftwatermark,transparent}
\SetWatermarkAngle{0}
\SetWatermarkText{\transparent{0.025}\includegraphics[scale=0.75]{graphics/logo_black.png}}

%Color-Coded questions
\newcommand{\ColourQuestion}[3]{\renewcommand{\questionlabel}{\colorbox{#1}{\bfseries\color{white}\thequestion}\hfill}\question[#3] #2}
\newcommand{\BlueQuestion}[1]{\ColourQuestion{RoyalBlue}{#1}{1}}
\newcommand{\GreenQuestion}[1]{\ColourQuestion{ForestGreen}{#1}{2}}
\newcommand{\YellowQuestion}[1]{\ColourQuestion{Goldenrod}{#1}{4}}
\newcommand{\RedQuestion}[1]{\ColourQuestion{BrickRed}{#1}{6}}
\newcommand{\PurpleQuestion}[1]{\ColourQuestion{RoyalPurple}{#1}{30}}
\pointsinrightmargin

\header{\footnotesize\scshape Assignment \#\AssignmentNumber: \AssignmentTitle}{}{}
\cfoot{\footnotesize\scshape \AssignmentCourse\\Woodstock School---Mussoorie, Uttarakhand---India}


\def\AssignmentCourse{AP Computer Science A}
\def\AssignmentNumber{12}
\def\AssignmentTitle{The Linked List}

\begin{document}
  \begin{questions}
    \BlueQuestion{Explain at least one benefit and one drawback for using a linked list in place of a traditional array in the \code{Stack} and \code{Queue} data structures.}

    \BlueQuestion{Implement the \code{trimN()} method in our \code{LinkedList} class. This method should take \code{N} as a parameter and remove the last \code{N} nodes from the list.}

    \GreenQuestion{Implement \code{Queue} using a linked list. Ensure you maintain the use of generics as appropriate.}

    \GreenQuestion{Implement the \code{find()} method in our \code{LinkedList} class. This method should take some data, \code{key}, as a parameter and return the index where \code{key} exists as the data for a particular \code{Node}. This method should return the value $-1$ if \code{key} does not exist in the linked list.}

    \YellowQuestion{\emph{Doubly-Linked List.} Make all necessary changes to our \code{LinkedList} in order to implement a \emph{doubly-linked list} in which each node has both \code{Next} and \code{Previous} attributes. In particular, ensure that \code{add()} and \code{remove()} work as intended.}

    \RedQuestion{\emph{Deque.} A double-ended queue or \emph{deque} (pronounced ``deck'') is like a stack or a queue but supports adding and removing items from both ends. A deque stores a collection of items and supports the following operations:}
      \begin{center}
        \small
        \begin{tabular}{p{0.33\textwidth} r}
          \code{isEmpty()} & is the deque empty?\\
          \code{size()} & how many items are in the deque?\\
          \code{pushLeft(item)} & add an item to the left end\\
          \code{pushRight(item)} & add an item to the right end\\
          \code{popLeft()} & remove and return an item from the left end\\
          \code{popRight()} & remove and return an item from the right end
        \end{tabular}
      \end{center}
    Implement a \emph{deque} in Java using a doubly-linked list.
  \end{questions}
\end{document}
