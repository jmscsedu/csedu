\documentclass[10pt]{exam}
\usepackage[T1]{fontenc}
\usepackage[paper=a4paper,margin=2cm]{geometry}
\usepackage[sfdefault,light]{roboto}

\usepackage[usenames,dvipsnames]{xcolor}
\usepackage{amsmath,amssymb,array,graphicx,enumitem,listings,lstautogobble,multicol,textcomp,titlesec}
\usepackage{mathtools}

\setlength\parindent{0cm}

%Title & Section Formatting
\titleformat{\section}{\normalfont\Large\bfseries}{}{0em}{}[{\titlerule[0.5pt]}]
\titleformat{\subsection}{\normalfont\large\bfseries}{}{0em}{}

%Mathematical Shortcuts
\newcommand{\floor}[1]{\left\lfloor #1 \right\rfloor}
\newcommand{\ceil}[1]{\left\lceil #1 \right\rceil}

%Listings Shortcuts & Settings
\newcommand{\code}[1]{\lstinline{#1}}
\lstset{language=Java,
        autogobble=true,
        basicstyle=\ttfamily,
        commentstyle=\color{black!45},
        keywordstyle=\bfseries,
        showstringspaces=false,
        upquote=true}

%General Shortcuts
\newcommand{\blankpage}{\null\thispagestyle{empty}\addtocounter{page}{-1}\newpage}

\usepackage{draftwatermark,transparent}
\SetWatermarkAngle{0}
\SetWatermarkText{\transparent{0.025}\includegraphics[scale=0.75]{graphics/logo_black.png}}

%Color-Coded questions
\newcommand{\ColourQuestion}[3]{\renewcommand{\questionlabel}{\colorbox{#1}{\bfseries\color{white}\thequestion}\hfill}\question[#3] #2}
\newcommand{\BlueQuestion}[1]{\ColourQuestion{RoyalBlue}{#1}{1}}
\newcommand{\GreenQuestion}[1]{\ColourQuestion{ForestGreen}{#1}{2}}
\newcommand{\YellowQuestion}[1]{\ColourQuestion{Goldenrod}{#1}{4}}
\newcommand{\RedQuestion}[1]{\ColourQuestion{BrickRed}{#1}{6}}
\newcommand{\PurpleQuestion}[1]{\ColourQuestion{RoyalPurple}{#1}{30}}
\pointsinrightmargin

\header{\footnotesize\scshape Assignment \#\AssignmentNumber: \AssignmentTitle}{}{}
\cfoot{\footnotesize\scshape \AssignmentCourse\\Woodstock School---Mussoorie, Uttarakhand---India}


\def\AssignmentCourse{AP Computer Science A}
\def\AssignmentNumber{09}
\def\AssignmentTitle{Computational Algorithms}

\begin{document}
  \begin{questions}
    \BlueQuestion{Under what circumstances might it be more appropriate to select an asymptotically ``slower'' sorting algorithm? Be as specific as possible.}

    \BlueQuestion{Which is ``asymptotically greater'', $n!$ or $n^n$? Explain your reasoning.}

    \GreenQuestion{Explain why the following program segment runs in $O(\lg n)$ time.}
      \begin{lstlisting}
        while (n > 0) {
          System.out.print(n % 2);
          n /= 2;
        }
      \end{lstlisting}

    \GreenQuestion{Explain why it is important to understand when a particular problem is algorithmically ``unsolvable.'' What can be done to tackle these types of problems?}

    \YellowQuestion{Consider \emph{Euclid's Algorithm} given below.}\\[4pt]
      \begin{minipage}{0.5\textwidth}
        Algorithm:\\
        \textbf{Input:} $a, b$: any two, positive integers.
        \begin{lstlisting}[mathescape=true]
          while $a \neq b$
            if $a > b$
              $a \leftarrow a - b$
            otherwise
              $b \leftarrow b - a$
          return $a$
        \end{lstlisting}
      \end{minipage}
      \begin{minipage}{0.5\textwidth}
        Example:\\
        \textbf{Input:} $12, 18$
        \begin{center}
          \begin{tabular}{|c | c |}
            \hline
            \qquad $\mathbf{a}$\qquad\ & \qquad $\mathbf{b}$\qquad\ \\
            \hline
            $12$ & $18$\\
            \hline
            $12$ & $6$\\
            \hline
            $6$ & $6$\\
            \hline
          \end{tabular}
        \end{center}
        \textbf{Output:} $6$
      \end{minipage}

      \begin{parts}
        \part Follow \emph{Euclid's Algorithm} for each of the following pairs of numbers.
          \begin{enumerate}[label=\roman*.]
            \item $40, 16$
            \item $75, 105$
            \item $156, 286$
          \end{enumerate}
        \part How is the result of \emph{Euclid's Algorithm} ``mathematically significant''? That is, what does this algorithm actually produce as a result?
        \part Implement \emph{Euclid's Algorithm} in Java.
      \end{parts}

      \RedQuestion{The \emph{Sieve of Eratosthenes} is an ancient algorithm designed for finding prime numbers below a given maximum value. It accomplishes this by marking as non-prime all multiples of each prime, starting with $2$. The following steps summarize how the sieve works.

        \begin{enumerate}
          \item Create a list of consecutive integers from $2$ through $n$: $[2, 3, 4, 5, ..., n]$.
          \item Initialize $p$ to $2$, the smallest prime number.
          \item Starting with $2p$, mark each multiple of $p$ less than or equal to $n$ as non-prime.
          \item Set $p$ to the next number greater than $p$ that is not marked as non-prime. If no such number exists, stop.
          \item When the algorithm stops, all numbers not marked non-prime are all the prime numbers below $n$.
        \end{enumerate}

        Create the method, \code{Eratosthenes()}, that will take $n$ as a parameter and use the \emph{Sieve of Eratosthenes} to find all prime numbers less than or equal to $n$. Verify your implementation by having your method print every found prime number.}

  \end{questions}
\end{document}
