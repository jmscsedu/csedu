\documentclass[10pt]{exam}
\usepackage[T1]{fontenc}
\usepackage[paper=a4paper,margin=2cm]{geometry}
\usepackage[sfdefault,light]{roboto}

\usepackage[usenames,dvipsnames]{xcolor}
\usepackage{amsmath,amssymb,array,graphicx,enumitem,listings,lstautogobble,multicol,textcomp,titlesec}
\usepackage{mathtools}

\setlength\parindent{0cm}

%Title & Section Formatting
\titleformat{\section}{\normalfont\Large\bfseries}{}{0em}{}[{\titlerule[0.5pt]}]
\titleformat{\subsection}{\normalfont\large\bfseries}{}{0em}{}

%Mathematical Shortcuts
\newcommand{\floor}[1]{\left\lfloor #1 \right\rfloor}
\newcommand{\ceil}[1]{\left\lceil #1 \right\rceil}

%Listings Shortcuts & Settings
\newcommand{\code}[1]{\lstinline{#1}}
\lstset{language=Java,
        autogobble=true,
        basicstyle=\ttfamily,
        commentstyle=\color{black!45},
        keywordstyle=\bfseries,
        showstringspaces=false,
        upquote=true}

%General Shortcuts
\newcommand{\blankpage}{\null\thispagestyle{empty}\addtocounter{page}{-1}\newpage}

\usepackage{draftwatermark,transparent}
\SetWatermarkAngle{0}
\SetWatermarkText{\transparent{0.025}\includegraphics[scale=0.75]{graphics/logo_black.png}}

%Color-Coded questions
\newcommand{\ColourQuestion}[3]{\renewcommand{\questionlabel}{\colorbox{#1}{\bfseries\color{white}\thequestion}\hfill}\question[#3] #2}
\newcommand{\BlueQuestion}[1]{\ColourQuestion{RoyalBlue}{#1}{1}}
\newcommand{\GreenQuestion}[1]{\ColourQuestion{ForestGreen}{#1}{2}}
\newcommand{\YellowQuestion}[1]{\ColourQuestion{Goldenrod}{#1}{4}}
\newcommand{\RedQuestion}[1]{\ColourQuestion{BrickRed}{#1}{6}}
\newcommand{\PurpleQuestion}[1]{\ColourQuestion{RoyalPurple}{#1}{30}}
\pointsinrightmargin

\header{\footnotesize\scshape Assignment \#\AssignmentNumber: \AssignmentTitle}{}{}
\cfoot{\footnotesize\scshape \AssignmentCourse\\Woodstock School---Mussoorie, Uttarakhand---India}


\def\AssignmentCourse{AP Computer Science A}
\def\AssignmentNumber{02}
\def\AssignmentTitle{Conditionals}

\begin{document}
  \begin{questions}
    \BlueQuestion{What (if anything) is wrong with each of the following statements?}
      \begin{parts}
        \part \code{if (a > b) then c = 0;}
        \part \code{if a > b \{ c = 0; \}}
        \part \code{if (a > b) c = 0;}
      \end{parts}

    \BlueQuestion{Write a program that will use \code{Math.random()} to simulate the flipping of a fair coin. Your program should print \code{Heads} or \code{Tails}.}

    \GreenQuestion{What values for boolean variables, \code{X} and \code{Y}, will result in the following statement being true?}
      \begin{center}
        \code{!(X && Y) || !(X || Y)}
      \end{center}
    Explain your reasoning.

    \GreenQuestion{Write the program, \code{Craps}, that will roll two $6$-sided dice and print the following:}
      \begin{itemize}
        \item \code{Snake Eyes} when two $1$'s are rolled,
        \item \code{Seven} when the dice values total $7$,
        \item \code{Boxcars} when two $6$'s are rolled,
        \item \code{Hard Way} when any other pair is rolled.
      \end{itemize}

    \YellowQuestion{The \emph{exclusive}-or (\emph{XOR}) boolean operator evaluates to \code{true} if exactly one of two boolean expressions is \code{true} and \code{false} otherwise.}
      \begin{parts}
        \part Complete the following truth table for the XOR operator ($\oplus$).
          \begin{center}
            \begin{tabular}{c | c | c }
              $\mathbf{P}$ & $\mathbf{Q}$ & $\mathbf{P} \oplus \mathbf{Q}$ \\
              \hline
              $T$ & $T$ & \ \\
              $T$ & $F$ & \ \\
              $F$ & $T$ & \ \\
              $F$ & $F$ & \ \\
            \end{tabular}
          \end{center}
        \part Write a boolean expression that will satisfy the output of $\mathbf{P} \oplus \mathbf{Q}$ above using any combination of AND (\code{\&\&}), OR (\code{||}) and NOT (\code{!}) logical operators.
      \end{parts}

    \RedQuestion{\emph{The Quadratic Formula.} Write the program, \code{Quadratic}, that, when given $a$, $b$, and $c$, the coefficients of a quadratic function in the form $y = ax^2 + bx + c$, correctly prints the roots of the function. Be sure to consider the following:}
      \begin{itemize}
        \item The quadratic formula returns the roots of a given quadratic function. It is as follows:
          \[
            x = \dfrac{-b \pm \sqrt{b^2 - 4ac}}{2a}
          \]
        {\small\textbf{Note:} The command \code{Math.sqrt(n)} will return the square root of the number, $n$.}

        \item When $a$ is $0$, in order to avoid dividing by zero when using the quadratic formula, you should instead solve the equation:
          \[
            bx + c = 0
          \]

        \item When the discriminant, $b^2 - 4ac$, is negative, there is no real solution. Your program should instead print: \code{No Real Solutions}.
      \end{itemize}
  \end{questions}
\end{document}
