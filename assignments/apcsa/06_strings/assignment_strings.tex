\documentclass[10pt]{exam}
\usepackage[T1]{fontenc}
\usepackage[paper=a4paper,margin=2cm]{geometry}
\usepackage[sfdefault,light]{roboto}

\usepackage[usenames,dvipsnames]{xcolor}
\usepackage{amsmath,amssymb,array,graphicx,enumitem,listings,lstautogobble,multicol,textcomp,titlesec}
\usepackage{mathtools}

\setlength\parindent{0cm}

%Title & Section Formatting
\titleformat{\section}{\normalfont\Large\bfseries}{}{0em}{}[{\titlerule[0.5pt]}]
\titleformat{\subsection}{\normalfont\large\bfseries}{}{0em}{}

%Mathematical Shortcuts
\newcommand{\floor}[1]{\left\lfloor #1 \right\rfloor}
\newcommand{\ceil}[1]{\left\lceil #1 \right\rceil}

%Listings Shortcuts & Settings
\newcommand{\code}[1]{\lstinline{#1}}
\lstset{language=Java,
        autogobble=true,
        basicstyle=\ttfamily,
        commentstyle=\color{black!45},
        keywordstyle=\bfseries,
        showstringspaces=false,
        upquote=true}

%General Shortcuts
\newcommand{\blankpage}{\null\thispagestyle{empty}\addtocounter{page}{-1}\newpage}


%Color-Coded questions
\newcommand{\ColourQuestion}[3]{\renewcommand{\questionlabel}{\colorbox{#1}{\bfseries\color{white}\thequestion}\hfill}\question[#3] #2}
\newcommand{\BlueQuestion}[1]{\ColourQuestion{RoyalBlue}{#1}{3}}
\newcommand{\GreenQuestion}[1]{\ColourQuestion{ForestGreen}{#1}{5}}
\newcommand{\YellowQuestion}[1]{\ColourQuestion{Goldenrod}{#1}{10}}
\newcommand{\RedQuestion}[1]{\ColourQuestion{BrickRed}{#1}{20}}
\newcommand{\PurpleQuestion}[1]{\ColourQuestion{RoyalPurple}{#1}{30}}
\pointsinrightmargin

\header{\footnotesize\scshape Assignment \#\AssignmentNumber: \AssignmentTitle}{}{}
\cfoot{\footnotesize\scshape \AssignmentCourse\\Woodstock School---Mussoorie, Uttarakhand---India}


\def\AssignmentCourse{AP Computer Science A}
\def\AssignmentNumber{06}
\def\AssignmentTitle{Strings}

\begin{document}
  \begin{questions}
    \BlueQuestion{What does the following code fragment print?}
      \begin{lstlisting}
        String string1 = "Hello";
        String string2 = string1;
        string1 = "World!";
        System.out.println(string1);
        System.out.println(string2);
      \end{lstlisting}
    \BlueQuestion{Write a brief explanation for the results in Question \#1.}
    \GreenQuestion{Given a string \code{site} that represents a website, write a code fragment to determine its top-level domain (TLD). For example, the TLD for the string \code{"http://www.woodstockschool.in"} is: \code{in}.}
    \GreenQuestion{A string, \code{s}, is a \emph{circular shift} of a string, \code{t}, if it matches when the characters are circularly shifted by any number of positions. For example, \code{ACTGACG} is a circular shift of \code{TGACGAC} and vice versa. Detecting this condition is important in the study of genomic sequences. Write a method that checks whether two given strings, \code{s} and \code{t}, are circular shifts of one another.}\\[4pt]{\small\textbf{Note:} This can be accomplished using a very simple technique involving string concatenation.}
    \YellowQuestion{\emph{Password Strength Verification.} Write a static method that takes a single, \code{String} argument and returns \code{true} if it meets all of the following conditions and \code{false} otherwise.}
      \begin{itemize}
        \item The string is at least $8$ characters in length.
        \item The string contains at least one numeric digit (0--9).
        \item The string contains at least one upper-case letter.
        \item The string contains at least one lower-case letter.
      \end{itemize}
    {\small\textbf{Hint:} There is a clever method for testing the last two requirements simultaneously.}
    \RedQuestion{\emph{Kama Sutra Cipher.} The \emph{Kama Sutra} describes a fairly simple encryption technique (listed as the ``art of secret writing''). It requires a one-to-one pairing of letters. A message can then be encoding by replacing every letter with its pair.}\\[4pt]
    {\small\textbf{Example:} Suppose that the following table of pairings were being used:
      \begin{center}
        \begin{tabular}{| c | c | c | c | c | c | c | c | c | c | c | c | c |}
          \hline
          T & H & E & Q & U & I & C & K & B & R & O & W & N\\
          \hline
          F & X & J & M & P & S & V & L & A & Z & Y & D & G\\
          \hline
        \end{tabular}
      \end{center}
    Then, the message ``MEET AFTER SCHOOL'' would be encoded as: ``QJJF BTFJZ IVYYK''.

    Often, the pairings will be recorded as a series of keywords which are then used to create the table of pairings. (In the above example, the two keywords are: ``THEQUICKBROWN'' and ``FXJMPSVLAZYDG''.)}

    Create a method, \code{KamaSutra()}, that will take three, \code{String} parameters: the first parameter is the text to be encoded and the remaining two parameters are the two keywords for a Kama Sutra cipher and returns the encrypted text.
  \end{questions}
\end{document}
