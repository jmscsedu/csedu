\documentclass[10pt]{exam}
\usepackage[T1]{fontenc}
\usepackage[paper=a4paper,margin=2cm]{geometry}
\usepackage[sfdefault,light]{roboto}

\usepackage[usenames,dvipsnames]{xcolor}
\usepackage{amsmath,amssymb,array,graphicx,enumitem,listings,lstautogobble,multicol,textcomp,titlesec}
\usepackage{mathtools}

\setlength\parindent{0cm}

%Title & Section Formatting
\titleformat{\section}{\normalfont\Large\bfseries}{}{0em}{}[{\titlerule[0.5pt]}]
\titleformat{\subsection}{\normalfont\large\bfseries}{}{0em}{}

%Mathematical Shortcuts
\newcommand{\floor}[1]{\left\lfloor #1 \right\rfloor}
\newcommand{\ceil}[1]{\left\lceil #1 \right\rceil}

%Listings Shortcuts & Settings
\newcommand{\code}[1]{\lstinline{#1}}
\lstset{language=Java,
        autogobble=true,
        basicstyle=\ttfamily,
        commentstyle=\color{black!45},
        keywordstyle=\bfseries,
        showstringspaces=false,
        upquote=true}

%General Shortcuts
\newcommand{\blankpage}{\null\thispagestyle{empty}\addtocounter{page}{-1}\newpage}

\usepackage{draftwatermark,transparent}
\SetWatermarkAngle{0}
\SetWatermarkText{\transparent{0.025}\includegraphics[scale=0.75]{graphics/logo_black.png}}

%Color-Coded questions
\newcommand{\ColourQuestion}[3]{\renewcommand{\questionlabel}{\colorbox{#1}{\bfseries\color{white}\thequestion}\hfill}\question[#3] #2}
\newcommand{\BlueQuestion}[1]{\ColourQuestion{RoyalBlue}{#1}{1}}
\newcommand{\GreenQuestion}[1]{\ColourQuestion{ForestGreen}{#1}{2}}
\newcommand{\YellowQuestion}[1]{\ColourQuestion{Goldenrod}{#1}{4}}
\newcommand{\RedQuestion}[1]{\ColourQuestion{BrickRed}{#1}{6}}
\newcommand{\PurpleQuestion}[1]{\ColourQuestion{RoyalPurple}{#1}{30}}
\pointsinrightmargin

\header{\footnotesize\scshape Assignment \#\AssignmentNumber: \AssignmentTitle}{}{}
\cfoot{\footnotesize\scshape \AssignmentCourse\\Woodstock School---Mussoorie, Uttarakhand---India}


\def\AssignmentCourse{AP Computer Science Principles}
\def\AssignmentNumber{06}
\def\AssignmentTitle{Active Mode}

\begin{document}
  \begin{questions}
    \BlueQuestion{If \code{dx} holds the horizontal speed a particular element should travel from frame to frame for a given program, write a code fragment to reverse its horizontal direction \emph{without} using a conditional.}

    \BlueQuestion{Explain why variables whose changes need to persist from frame to frame should be declared \emph{outside} the \code{draw()} method.}

    \GreenQuestion{The following code fragment produces an error. Why do you think Processing prohibits you from creating a program with a mixture of ``active'' and ``static'' mode elements?}
      \begin{lstlisting}
        void setup() {
          size(250, 250);
          background(255);
        }

        ellipse(125, 125, 50, 50);
      \end{lstlisting} 

    \GreenQuestion{Write a program that will draw a small circle in the center of the window and gradually increase its size over time.}

    \YellowQuestion{Write a program that will draw a $10 \times 10$ box in the upper left-hand corner of the window and animate the box moving along the window's edge.}\\
      {\small\textbf{Note:} Your program should work regardless of window size and should represent an ``infinite'' loop. That is, the box should continue travelling around the edge of the window until the program is closed.}

    \RedQuestion{The following program is designed as a very simple simulation of gravity.} %Gravity (Help + Questions)
      \begin{lstlisting}[basicstyle=\small\ttfamily]
        float ballHeight = 0;     // starting height of the ball
        float speed = 0, g = 0.1; // current speed of the ball and acceleration due to gravity
        float dampening = -0.8;   // dampening effect for the ball

        void setup() {
          size(100, 250);
        }

        void draw() {
          // clear the screen
          background(255);
          // draw the ball
          fill(0);
          ellipse(width / 2, ballHeight, 10, 10);

          // modify the ball's height
          ballHeight += speed;
          // increase the speed of the ball due to gravity
          speed += g;

          // bounce!
          if (ballHeight > height) {
            speed *= dampening;
            ballHeight = height;
          }
        }
      \end{lstlisting}
      Run the program a few times with different values for \code{g} and \code{dampening} and answer each of the following questions.
      \begin{parts}
        \part Explain what effect \code{dampening} has on the ball. 
        \part What would happen if \code{dampening} were set to a positive value? To $-1$? Explain each of these.
        \part Explain why \code{g} is set to a fairly small number. What happens if it is set to a larger number, such as $5$?        
        \part Explain why the ball never comes to a \code{rest}. How would you fix this issue?
      \end{parts}
  \end{questions}
\end{document}
