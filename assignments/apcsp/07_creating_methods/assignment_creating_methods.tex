\documentclass[10pt]{exam}
\usepackage[T1]{fontenc}
\usepackage[paper=a4paper,margin=2cm]{geometry}
\usepackage[sfdefault,light]{roboto}

\usepackage[usenames,dvipsnames]{xcolor}
\usepackage{amsmath,amssymb,array,graphicx,enumitem,listings,lstautogobble,multicol,textcomp,titlesec}
\usepackage{mathtools}

\setlength\parindent{0cm}

%Title & Section Formatting
\titleformat{\section}{\normalfont\Large\bfseries}{}{0em}{}[{\titlerule[0.5pt]}]
\titleformat{\subsection}{\normalfont\large\bfseries}{}{0em}{}

%Mathematical Shortcuts
\newcommand{\floor}[1]{\left\lfloor #1 \right\rfloor}
\newcommand{\ceil}[1]{\left\lceil #1 \right\rceil}

%Listings Shortcuts & Settings
\newcommand{\code}[1]{\lstinline{#1}}
\lstset{language=Java,
        autogobble=true,
        basicstyle=\ttfamily,
        commentstyle=\color{black!45},
        keywordstyle=\bfseries,
        showstringspaces=false,
        upquote=true}

%General Shortcuts
\newcommand{\blankpage}{\null\thispagestyle{empty}\addtocounter{page}{-1}\newpage}

\usepackage{draftwatermark,transparent}
\SetWatermarkAngle{0}
\SetWatermarkText{\transparent{0.025}\includegraphics[scale=0.75]{graphics/logo_black.png}}

%Color-Coded questions
\newcommand{\ColourQuestion}[3]{\renewcommand{\questionlabel}{\colorbox{#1}{\bfseries\color{white}\thequestion}\hfill}\question[#3] #2}
\newcommand{\BlueQuestion}[1]{\ColourQuestion{RoyalBlue}{#1}{1}}
\newcommand{\GreenQuestion}[1]{\ColourQuestion{ForestGreen}{#1}{2}}
\newcommand{\YellowQuestion}[1]{\ColourQuestion{Goldenrod}{#1}{4}}
\newcommand{\RedQuestion}[1]{\ColourQuestion{BrickRed}{#1}{6}}
\newcommand{\PurpleQuestion}[1]{\ColourQuestion{RoyalPurple}{#1}{30}}
\pointsinrightmargin

\header{\footnotesize\scshape Assignment \#\AssignmentNumber: \AssignmentTitle}{}{}
\cfoot{\footnotesize\scshape \AssignmentCourse\\Woodstock School---Mussoorie, Uttarakhand---India}


\def\AssignmentCourse{AP Computer Science Principles}
\def\AssignmentNumber{07}
\def\AssignmentTitle{Creating Methods}

\begin{document}
  \begin{questions}
    \BlueQuestion{Identify the syntax error in each of the following.} %Syntax Error
      \begin{parts}
        \part \begin{lstlisting}
          void drawIt(x, y) {
            // implementation not shown
          }
        \end{lstlisting}

        \part \begin{lstlisting}
          void doSomething {
            // implementation not shown
          }
        \end{lstlisting}

        \part \begin{lstlisting}
          mystery(boolean a) {
            // implementation not shown
          }
        \end{lstlisting}
      \end{parts}

    \BlueQuestion{Explain why the following does not work as intended.} %Variable Scope Error
      \begin{lstlisting}
        int a = 4;

        void draw() {
          incr(a);
        }

        void incr(int a) {
          a = a + 1;
        }
      \end{lstlisting}

    \GreenQuestion{Explain why you might want to use a method to represent a section of your programming code, even if that section does not need to be repeated.} %Explanation on a method even if used only once...

    \GreenQuestion{Create the \code{dartboard()} method which will take as its sole parameter, \code{scale}, to use as the scale of the dartboard. Use your most recent rendition of the Dartboard program from Assignment \#5 as the basis for this method.}

    \YellowQuestion{
      Revisit your most recent Processing Bee implementation (Assignment \#3) to implement each of the following.
      \begin{parts}
        \part The \code{drawBee()} method which accepts the parameters, \code{x} and \code{y}, and draw the Processing Bee centered on the given \code{x} and \code{y}.
        \part An overloaded \code{drawBee()} method which accepts the parameters, \code{x} and \code{y}, and additionally accepts the \code{scale} parameter to draw the Processing Bee centered on the given \code{x} and \code{y} and scaled according to the given scale factor.
        \part Change your solution to part (a) to call your method from part (b). Explain why this approach might be beneficial.
      \end{parts}
    }

    \RedQuestion{
      Implement the \code{transformAndDrawPoint()} method with the following definition:
      \begin{center}
        \code{transformAndDrawPoint(x, y, xMin, yMin, xMax, yMax, size)}
      \end{center}
      which will transform the point, $(x, y)$, from a coordinate plane defined by $(xMin, yMin), (xMax, yMax)$ to one defined by $(0, 0), (width, height)$ and draw it on the screen as a circle centered at $(x^\prime, y^\prime)$ with diameter $size$.
    }\\
    {\small\textbf{Note:} You may want to use your solution to drawing a graph from Assignment \#5 in order to guide and test your solution for this question.}
  \end{questions}
\end{document}
