\documentclass[10pt]{exam}
\usepackage[T1]{fontenc}
\usepackage[paper=a4paper,margin=2cm]{geometry}
\usepackage[sfdefault,light]{roboto}

\usepackage[usenames,dvipsnames]{xcolor}
\usepackage{amsmath,amssymb,array,graphicx,enumitem,listings,lstautogobble,multicol,textcomp,titlesec}
\usepackage{mathtools}

\setlength\parindent{0cm}

%Title & Section Formatting
\titleformat{\section}{\normalfont\Large\bfseries}{}{0em}{}[{\titlerule[0.5pt]}]
\titleformat{\subsection}{\normalfont\large\bfseries}{}{0em}{}

%Mathematical Shortcuts
\newcommand{\floor}[1]{\left\lfloor #1 \right\rfloor}
\newcommand{\ceil}[1]{\left\lceil #1 \right\rceil}

%Listings Shortcuts & Settings
\newcommand{\code}[1]{\lstinline{#1}}
\lstset{language=Java,
        autogobble=true,
        basicstyle=\ttfamily,
        commentstyle=\color{black!45},
        keywordstyle=\bfseries,
        showstringspaces=false,
        upquote=true}

%General Shortcuts
\newcommand{\blankpage}{\null\thispagestyle{empty}\addtocounter{page}{-1}\newpage}


%Color-Coded questions
\newcommand{\ColourQuestion}[3]{\renewcommand{\questionlabel}{\colorbox{#1}{\bfseries\color{white}\thequestion}\hfill}\question[#3] #2}
\newcommand{\BlueQuestion}[1]{\ColourQuestion{RoyalBlue}{#1}{3}}
\newcommand{\GreenQuestion}[1]{\ColourQuestion{ForestGreen}{#1}{5}}
\newcommand{\YellowQuestion}[1]{\ColourQuestion{Goldenrod}{#1}{10}}
\newcommand{\RedQuestion}[1]{\ColourQuestion{BrickRed}{#1}{20}}
\newcommand{\PurpleQuestion}[1]{\ColourQuestion{RoyalPurple}{#1}{30}}
\pointsinrightmargin

\header{\footnotesize\scshape Assignment \#\AssignmentNumber: \AssignmentTitle}{}{}
\cfoot{\footnotesize\scshape \AssignmentCourse\\Woodstock School---Mussoorie, Uttarakhand---India}

\usepackage{textcomp}

\def\AssignmentCourse{AP Computer Science Principles}
\def\AssignmentNumber{03}
\def\AssignmentTitle{Using Variables}

\begin{document}
  \begin{questions}
    %Random
    \BlueQuestion{The method call, \code{random(256)} will generate a random value between $0$ and $255$. Create a Processing program that will generate and store three random values for red, green, and blue and draw a circle using this random colour as its fill. Run your program multiple times to observe the random colour effect.}

    %Explanation as to initialization purpose?
    \BlueQuestion{Explain which data type you'd use to store each of the following pieces of information.}
      \begin{parts}
        \part The year you are graduating from high school,
        \part The area of a given triangle,
        \part Whether or not you are taking a Computer Science course.
      \end{parts}

    \GreenQuestion{Use the following code fragment to answer each of the following questions.}
      \begin{center}
        \code{int n = 50.7;}
      \end{center}
      \begin{parts}
        \part What error is reported when you run this code fragment?
        \part On the other hand, a statement like: \code{float n = 50;} runs without an error. Why do you think initializing an integer variable to a float causes an error while doing the reverse does not?
      \end{parts}

    \GreenQuestion{The \code{int()} method will accept a non-integer parameter and attempt to convert it into an integer. For instance, \code{int(4.7)} will return the integer value of \code{4}. Create a Processing sketch to \code{print()} the result of \code{int('a')}. What do you think the printed value indicates? Print a number of other characters to test your claim.}

    %Modify size of dartboard?
    \YellowQuestion{Using a \code{float} variable, \code{dartBoardScale}, modify your dartboard drawing program from Assignment \#2 so that it will scale by a factor of \code{dartBoardScale}. That is, \code{dartBoardScale=0.5} will produce a dartboard half the size of the original, while \code{dartBoardScale=2.0} will produce one double in size.}

    \RedQuestion{Complete each of the following exercises.}
      \begin{parts}
        \part Use two integer variables, \code{beeX} and \code{beeY}, to represent the center of the Processing Bee's head. Modify all other $x$ and $y$ values in your Processing Bee program to reference \code{beeX} and \code{beeY}.\\
          {\small\textbf{Note:} Use the Processing Bee program you created in Assignment \#2 as the basis for this, but keep your old program. You'll need it for the next question.}
        \part Use the Processing reference for the \code{translate()} method (https://processing.org/reference/translate\_.html) to accomplish the same task as in part (a).
        \part How does your implementation differ from what Processing does with the \code{translate()} method? What potential pros and cons exist for using each of these methods?
      \end{parts}

  \end{questions}
\end{document}
