\documentclass[10pt]{exam}
\usepackage[T1]{fontenc}
\usepackage[paper=a4paper,margin=2cm]{geometry}
\usepackage[sfdefault,light]{roboto}

\usepackage[usenames,dvipsnames]{xcolor}
\usepackage{amsmath,amssymb,array,graphicx,enumitem,listings,lstautogobble,multicol,textcomp,titlesec}
\usepackage{mathtools}

\setlength\parindent{0cm}

%Title & Section Formatting
\titleformat{\section}{\normalfont\Large\bfseries}{}{0em}{}[{\titlerule[0.5pt]}]
\titleformat{\subsection}{\normalfont\large\bfseries}{}{0em}{}

%Mathematical Shortcuts
\newcommand{\floor}[1]{\left\lfloor #1 \right\rfloor}
\newcommand{\ceil}[1]{\left\lceil #1 \right\rceil}

%Listings Shortcuts & Settings
\newcommand{\code}[1]{\lstinline{#1}}
\lstset{language=Java,
        autogobble=true,
        basicstyle=\ttfamily,
        commentstyle=\color{black!45},
        keywordstyle=\bfseries,
        showstringspaces=false,
        upquote=true}

%General Shortcuts
\newcommand{\blankpage}{\null\thispagestyle{empty}\addtocounter{page}{-1}\newpage}


%Color-Coded questions
\newcommand{\ColourQuestion}[3]{\renewcommand{\questionlabel}{\colorbox{#1}{\bfseries\color{white}\thequestion}\hfill}\question[#3] #2}
\newcommand{\BlueQuestion}[1]{\ColourQuestion{RoyalBlue}{#1}{3}}
\newcommand{\GreenQuestion}[1]{\ColourQuestion{ForestGreen}{#1}{5}}
\newcommand{\YellowQuestion}[1]{\ColourQuestion{Goldenrod}{#1}{10}}
\newcommand{\RedQuestion}[1]{\ColourQuestion{BrickRed}{#1}{20}}
\newcommand{\PurpleQuestion}[1]{\ColourQuestion{RoyalPurple}{#1}{30}}
\pointsinrightmargin

\header{\footnotesize\scshape Assignment \#\AssignmentNumber: \AssignmentTitle}{}{}
\cfoot{\footnotesize\scshape \AssignmentCourse\\Woodstock School---Mussoorie, Uttarakhand---India}


\def\AssignmentCourse{AP Computer Science Principles}
\def\AssignmentNumber{01}
\def\AssignmentTitle{Introduction to Processing}

\begin{document}
  \begin{questions}
    %Errors
    \BlueQuestion{Why do you think \code{size()} needs to be placed before any other drawing methods? What error do you get if you don't place it first?}

    \BlueQuestion{The keywords \code{width} and \code{height} evaluate to the width and height, respectively, of the drawing canvas. Use these keywords to draw an ellipse exactly centered on the canvas.\\
    {\small\textbf{Note:} Your program \emph{must} use \code{width} and \code{height} in order to work dynamically for differently sized canvases.}}

    %Fix Syntax
    \GreenQuestion{Identify and fix the errors in each of the following.}
      \begin{parts}
        \part \code{size(200, 200;}
        \part \code{background();}
        \part \code{stroke 255;}
        \part \code{fill(150)}
        \part \code{rectMode(center);}
      \end{parts}

    %Basic Drawing
    \GreenQuestion{Draw the following sets of concentric circles and squares.}
    \begin{center}
      \includegraphics[scale=0.6]{files/Concentric}
    \end{center}

    %Emulate a Drawing
    \RedQuestion{Use the Processing reference documents for \code{quad()} (https://processing.org/reference/quad\_.html) in order to draw the Processing Bee.}
      \begin{center}
        \includegraphics[scale=0.6]{files/ProcessingBeeGrayscale}
      \end{center}

    \YellowQuestion{Explain the iterative steps you took to create your Processing Bee program.}
  \end{questions}
\end{document}
